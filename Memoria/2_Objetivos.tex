\chapter{Objetivos}
\label{chap:objetivos}

\drop{E}{n} el siguiente capítulo se presentan los objetivos que se marcaron para el desarrollo de
este \acf{TFG} desde la realización del Anteproyecto de forma general y específica. De esta manera,
se determina el alcance del proyecto y los resultados que se esperan tras su implementación.

\section{Objetivo general}

Incentivado por el gran número de accidentes que se producen continuamente por distracciones a la
hora de utilizar los navegadores vía satélite actuales, en este trabajo se desarrollará un sistema
de navegación por satélite que haga uso de la interacción implícita, es decir, un sistema de
navegación que nos avise de las acciones a realizar sin necesidad de prestarle atención.

Dado que la única interacción implícita que ofrecen los navegadores actuales es el sonido, los
ciclistas y motoristas no pueden hacer uso de la navegación por satélite sin incrementar
considerablemente el riesgo de accidente. Esto es debido a que mirar la pantalla del navegador
provoca el 26\% de los accidentes producidos por distracciones \cite{Allianz14}, resulta complicado
oír nuestro dispositivo en ambientes abiertos y el uso de cascos o auriculares esta sancionado en
España \cite{Serrano14}.

Por qué vibración...

Con la finalidad de hacer más segura la navegación para los ciclistas y motoristas, y para facilitar
su uso para los peatones, el objetivo del \acf{TFG} es el «Desarrollo de un sistema de navegación
  por satélite que se comunique con el usuario por medio de la vibración de alguno de sus
  componentes»

\section{Objetivos específicos}

A partir del objetivo principal descrito en la sección anterior se determinan los objetivos
específicos en los siguientes apartados.

\subsection{A}


Puesto que la iteración con los navegadores vía satélite actuales se basan en el contacto
visual o mensajes auditivos, el trabajo consistirá en incorporar un nuevo tipo de
interacción implícita: la vibración. Por tanto, el objetivo que se pretende conseguir con
el trabajo es:

\begin{itemize}
\item Desarrollo de un sistema de navegación vía satélite que se
  comunique con el usuario por medio de la vibración de alguno de sus componentes
\end{itemize}

Este objetivo principal puede desglosarse en varios objetivos específicos:

\begin{itemize}
\item Estudio de las alternativas para interacción implícita
\item Desarrollo de la aplicación de navegación para smartphone
\item Búsqueda, selección e integración de de los complementos adecuados para
  interacción implícita
\end{itemize}



% Local Variables:
% TeX-master: "main.tex"
%  coding: utf-8
%  mode: latex
%  mode: flyspell
%  ispell-local-dictionary: "castellano8"
% End:
