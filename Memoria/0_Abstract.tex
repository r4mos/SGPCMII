\chapter{Abstract}

Not too long ago it was incredible to carry a tool in our pocket that would allow us to know our
exact location. It was much less credible to imagine that this tool would give us the ability to be
always connected to the Internet. However, today that tool exists and it is called a
\emph{smartphone}.

With the recent popularity of smartphones have emerged a long series of \emph{wearables} accessories
that communicate with \emph{smartphones} and provide us a new form of interaction with the device.

In this project it is shown one of the most important problems related to the use of
\emph{smartphones} for satellite navigation: ease of cause a distraction, and many times an
accident. To solve this it is proposed a new form of interaction in which the system is responsible
for warning the user, when he has to perform some action through \emph{wearables}. In this way we
avoid the user to have to look at the screen and will prevent this type of accident.

The application developed throughout the project is called \emph{Naviganto} and it offers the user
the possibility to surf via satellite guided by the vibrations of two devices: one placed on the
left side of the body, and the other one located on its right side. Moreover, \emph{Naviganto}
provides the user with common interactions of navigation applications such as the screen or the
sound.

% Local Variables:
% TeX-master: "main.tex"
%  coding: utf-8
%  mode: latex
% End:
