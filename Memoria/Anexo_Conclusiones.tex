\chapter{Conclusión personal}
\label{chap:conclusionesPersonales}

El desarrollo de este \acs{TFG} pone fin a mis años como estudiante universitario. A lo largo de
este tiempo, he adquirido numerosos conocimientos que me han servido para llevar a buen puerto este
desarrollo. Sin embargo, también he tenido que enfrentarme a tecnologías y formas de programar que
no conocía como es el desarrollo para Android y, especialmente, Android Wear. Esto me ha hecho
descubrir mi capacidad de adaptación a nuevas tecnologías, ver la repercusión de realizar
desarrollos eficientes en entornos con recursos reducidos y comprender la importancia de ejecutar en
diferentes \emph{threads} la \emph{interfaz} gráfica y las peticiones a Internet.

Como pienso que el mercado de aplicaciones móviles Android y Android Wear se encuentra en plena
expansión, considero que este proyecto me ha permitido conseguir conocimientos que me serán útiles
en un futuro no muy lejano. 

Por otro lado, con el desarrollo de este proyecto he podido comprobar que los desarrollos en
Informática permiten ver resultados muy rápidamente a costes muy bajos o prácticamente
nulos. Además, me ha permitido tomar en consideración lo aprendido durante la carrera para verificar
que es sólo una ínfima parte de lo que hay fuera. Estos hechos, han reforzado mis ganas de continuar
aprendiendo más sobre el ámbito de la Informática.

Por todo ello, considero que este proyecto se ha ajustado a lo que yo buscaba, y es el trabajo que
más he disfrutado haciendo en toda la carrera.

% Local Variables:
% TeX-master: "main.tex"
%  coding: utf-8
%  mode: latex
%  mode: flyspell
%  ispell-local-dictionary: "castellano8"
% End:
