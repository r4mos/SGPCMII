\chapter{Conclusiones}
\label{chap:conclusiones}

\drop{E}{n} el actual capítulo se presentan las conclusiones extraídas del desarrollo del
\acs{TFG}. Se comienza hablando de cómo se han alcanzado los diferentes objetivos, se continúa
describiendo posibles usos del sistema en entornos diferentes a los propuestos inicialmente; y se
termina enumerando posibles líneas de trabajo futuro que ampliarían la funcionalidad del presente
proyecto.

\section{Objetivos cumplidos}

Tras finalizar el desarrollo del trabajo propuesto, se puede considerar que se han alcanzado
satisfactoriamente los \textbf{objetivos específicos} planteados en el
Capítulo~\ref{chap:objetivos}:

\begin{itemize}
  \item El \textbf{desarrollo del sistema de posicionamiento} se abordó en las iteraciones 1 y 2
    (ver sección~\ref{sec:ite1} y sección~\ref{sec:ite2}). Tras su implementación el sistema fue
    capaz de mostrar la posición del usuario representada en un mapa.

  \item El \textbf{desarrollo del sistema de navegación} se abordó en las iteraciones 4, 5, 6 y 7
    (ver sección~\ref{sec:ite4}, sección~\ref{sec:ite5}, sección~\ref{sec:ite6} y
    sección\ref{sec:ite7}). Tras su implementación el sistema fue capaz de navegar desde la posición
    actual de usuario hasta la localización deseada.

  \item El \textbf{desarrollo del sistema de reproducción de audio} se abordó en la iteración 8 (ver
    sección~\ref{sec:ite8}). Tras su implementación fue posible navegar por una ruta siguiendo las
    indicaciones sonoras del sistema.

  \item El \textbf{desarrollo del sistema de comunicación con otros dispositivos} se abordó en la
    iteración 9 (ver sección~\ref{sec:ite9}). Tras su implementación fue posible intercambiar
    mensajes entre los diferentes dispositivos del sistema.

  \item El \textbf{desarrollo del sistema vibratorio} se abordó en la iteración 9 (ver
    sección~\ref{sec:ite9}). Tras su implementación fue posible hacer vibrar todos los componentes
    del sistema a las frecuencias oportunas.

\end{itemize}

Por ello, se puede concluir que se ha culminado el \textbf{objetivo general} del proyecto:
«desarrollo de un sistema de navegación por satélite que se comunique con el usuario por medio de la
  vibración de alguno de sus componentes».

\section{Otros usos del sistema}

Al tratarse de un trabajo dirigido a un público específico: peatones, ciclistas y motoristas; podría
suponerse que son los únicos que podrían verse beneficiados del mismo, pero no es así.

Por un lado, los conductores de cualquier tipo de vehículo motorizado (coches, camiones, etc)
podrían usar el sistema desarrollado en este trabajo para realizar sus rutas. \emph{Naviganto}
supone una alternativa real a las aplicaciones de navegación para \emph{smpartphones} actuales
siendo la única que puede comunicarse por vibraciones además de por la pantalla y el sonido.

Por otro lado, las personas que sufran algún tipo de discapacidad visual que provoque una
disminución de su visión podrá utilizar \emph{Naviganto} para sus desplazamientos, especialmente los
discapacitados con la visión de largo alcance. Puesto que el sistema no se haya desarrollado
específicamente para ellos, sólo podrían tener dificultad en la configuración inicial de los
dispositivos vibratorios y en la selección del destino. Estas tareas pueden hacerlas terceras
personas ya que sólo deben ejecutarse al comienzo de la ruta.

\section{Líneas de trabajo futuro}

A pesar de haber terminado el proyecto cumpliendo los objetivos propuestos inicialmente, hay
aspectos del sistema que podrían ser pulidos para mejorarlo en general o para ampliar la cuota de
usuarios finales del mismo. Estas futuras líneas de trabajo podrían ser:

\begin{definitionlist}
  \item[Integración para discapacitados visuales] Tal y como se mencionó en la sección anterior, el
    sistema desarrollado en este proyecto podría usarse por discapacitados visuales para sus
    desplazamientos. Para que no sea necesaria una tercera persona que configure los vibradores y
    seleccione el destino de la ruta, se podría incluir un módulo de reconocimiento de voz que
    sustituya todas las posibles acciones que se realizan en \emph{Naviganto}.

  \item[Implementación de funciones actuales sin conexión a Internet] A pesar de que a día de hoy es
    difícil encontrar personas que no dispongan de conexión a Internet en su \emph{smartphone}, es
    posible que en algunos momentos no podamos hacer uso de dicha conexión: ausencia de cobertura,
    viajes a otros países, etc. Para estas circunstancias sería interesante tener en nuestra
    aplicación los mapas del lugar, algunas instrucciones sonoras y algunas rutas precalculadas.

  \item[Implementación de nuevas funcionalidades] Entre las nuevas funcionalidades que se podrían
    incluir en \emph{Naviganto} podemos destacar dos que ya existen en otras aplicaciones de
    navegación para vehículos como:

    \begin{itemize}
      \item Representación visual del límite de velocidad de la vía en función del vehículo que se
        utilice para el desplazamiento e implementar una forma de avisar al usuario cuando sobrepase
        dicho límite.
      \item Representación visual del estado del tráfico de la vía por la que se circule.
    \end{itemize}

    Además, resultaría interesante integrar dentro del sistema una funcionalidad que no se ha podido
    encontrar en el resto de aplicaciones estudiadas en la sección~\ref{sec:apps}: aviso de radares
    fijos para los desplazamientos con vehículos motorizados. Puesto que la información de la
    localización de los radares es
    pública~\footnote{http://www.dgt.es/es/el-trafico/control-de-velocidad/} y es legal avisar de
    dichos radares en España~\cite{Arias14}, resultaría una funcionalidad muy interesante y fácil de
    integrar con avisos vibratorios.

  \item[Desarrollo para otras plataformas] En la actualidad, sólo tendría sentido portar el sistema
    desarrollado sobre Android con un 81\% de cuota de mercado a su principal competidor: iOS
    con el 12,9\% de cuota de mercado.

    Para que esta portabilidad tuviera éxito habría que modificar la forma en la que el sistema se
    comunica con los \emph{wearables} (ya que en iOS no estarían disponibles los \emph{Play
      Services}) u olvidarse de los \emph{wearables} con Android Wear, pero no de los
    \emph{wearables}. Aunque en la actualidad solamente los dispositivos con Android Wear permiten
    la manipulación a voluntad de su vibrador (ver sección~\ref{sec:wearables}), a lo largo de 2015
    Apple presentará su propio \emph{smartwatch} con su propio \acs{SO}: Watch
    OS~\footnote{https://www.apple.com/es/watch/overview/}. Como se espera que este \acs{SO} sea el
    competidor de Android Wear, también se espera que permita manipular el vibrador del dispositivo.

\end{definitionlist} 

% Local Variables:
% TeX-master: "main.tex"
%  coding: utf-8
%  mode: latex
%  mode: flyspell
%  ispell-local-dictionary: "castellano8"
% End:
