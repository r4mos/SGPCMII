\documentclass{pre-tfg}
\usepackage{longtable}

\title{Sistema de Guiado para Peatones, Ciclistas y Motoristas con Interacción Implícita}
\author{Carlos Ramos Mellado}
\advisorFirst{David Villa Alises}
\advisorDepartment{DEPARTAMENTO DE TECNOLOGÍAS Y SISTEMAS DE INFORMACIÓN}
\advisorSecond{}
\intensification{TECNOLOGÍAS DE LA INFORMACIÓN}
\docdate{2014}{Septiembre}


\begin{document}

\maketitle
\tableofcontents

\newpage

\section{INTRODUCCIÓN}

El guiado de personas es una actividad que se lleva desarrollando desde hace siglos con la ayuda de las estrellas, mapas y/o brújulas. Afortunadamente para nosotros, hoy en día es más fácil seguir un determinado camino gracias a la tecnología \textit{GPS}\footnote{https://es.wikipedia.org/wiki/Sistema\_de\_posicionamiento\_global}.

Con la popularización de los smartphones es común encontrar personas utilizando \textit{GPS} por medio de aplicaciones como \textit{Google Maps}\footnote{https://play.google.com/store/apps/details?id=com.google.android.apps.maps} o \textit{Sygic}\footnote{https://play.google.com/store/apps/details?id=com.sygic.aura} con el objetivo de desplazarse a algún lugar. Estas aplicaciones son realmente buenas para la navegación con coche pero tienen un problema: es necesario ver la pantalla o oír la las instrucciones para llevarlas a cabo. Para los peatones, ciclistas y motoristas el hecho de mirar a la pantalla o intentar oír el smartphone supone una distracción que puede terminar provocando un accidente y, por tanto, necesitan otro tipo de interacción con el dispositivo.

En este trabajo se desarrollará un sistema de guiado con interacción implícita, es decir, un sistema que guíe sin necesidad de estar pendiente de el. Para ello

%descripción
%imagen explicativa

%Además, puesto que el usuario no necesita interactuar con la pantalla podremos tenerla apagada y conseguir un notorio ahorro de batería.

\newpage

\section{TECNOLOGÍA ESPECÍFICA / INTENSIFICACIÓN / ITINERARIO CURSADO POR EL ALUMNO}

En la Tabla \ref{tab:tec-especifica} se muestra la tecnología especifica cursada por el alumno del Grado en Ingeniería Informática.

\begin{table}[hp]
  \centering
  \caption{Tecnología Específica cursada por el alumno}
  \label{tab:tec-especifica}

  \zebrarows{1}
  \begin{tabular}{l p{0.4\textwidth}}
    \multicolumn{2}{c}{\textbf{Marcar la tecnología cursada}} \\
    \hline
    X & Tecnologías de la Información \\
    ~ & Computación                   \\
    ~ & Ingeniería del Software       \\
    ~ & Ingeniería de Computadores    \\ \hline
  \end{tabular}
\end{table}

Del mismo modo, en la Tabla \ref{tab:competencias} se enumeran las competencias más destacables de la intensificación cursada y la justificación de uso de esa competencia concreta dentro del proyecto.

\rowcolors{2}{tabrowbg}{}
\begin{spacing}{1.0}
\begin{longtable}{p{0.45\linewidth}p{0.45\linewidth}}
  \caption{Justificación de las competencias específicas abordadas en el TFG}
  \label{tab:competencias} \\

  \textbf{Competencia} & \textbf{Justificación} \\
  \hline
  \hline
    Capacidad para comprender el entorno de una organización y sus necesidades en el ámbito de las tecnologías de la información y las comunicaciones. & \\

    Capacidad para seleccionar, diseñar, desplegar, integrar, evaluar, construir, gestionar, explotar y mantener las tecnologías de hardware, software y redes, dentro de los parámetros de coste y calidad adecuados. & \\

    Capacidad para emplear metodologías centradas en el usuario y la organización para el desarrollo, evaluación y gestión de aplicaciones y sistemas basados en tecnologías de la información que aseguren la accesibilidad, ergonomía y usabilidad de los sistemas. & \\

    Capacidad para seleccionar, diseñar, desplegar, integrar y gestionar redes e infraestructuras de comunicaciones en una organización. & \\

    Capacidad para seleccionar, desplegar, integrar y gestionar sistemas de información que satisfagan las necesidades de la organización, con los criterios de coste y calidad identificados. & \\

    Capacidad de concebir sistemas, aplicaciones y servicios basados en tecnologías de red, incluyendo Internet, web, comercio electrónico, multimedia, servicios interactivos y computación móvil. & \\

    Capacidad para comprender, aplicar y gestionar la garantía y seguridad de los sistemas informáticos. & \\

  \hline
\end{longtable}
\end{spacing}

\newpage

\section{OBJETIVOS}

De acuerdo a la Introducción, el alumno deberá especificar cuál o cuáles son las hipótesis
de trabajo de las que se parten, qué se pretende resolver, y en base a eso formular el
objetivo principal del TFG.

El objetivo principal deberá desglosarse en sub-objetivos parciales. Los sub-objetivos
deberán describirse de forma breve y concisa.

Como preámbulo a la formulación del objetivo parcial, el alumno deberá discutir sobre las
limitaciones y condicionantes a tener en cuenta en el desarrollo del TFG (lenguaje de
desarrollo, equipos, madurez de la tecnología, etc.).

Del mismo modo, será recomendable incluir una lista preliminar de requisitos del sistema a
construir.

\newpage

\section{MÉTODO Y FASES DE TRABAJO}

Para el desarrollo del proyecto, el alumno deberá seguir algún proceso o metodología afín
al problema que pretende resolver. Para ello, deberá aportar una pequeña descripción del
proceso o metodología (no más de una página) y \textbf{justificar su adecuación al
  problema a resolver}.

Del mismo modo, el alumno podrá realizar una breve planificación de la ejecución del
proyecto según el proceso o metodología seleccionada.

Como parte de la descripción del método y las fases de trabajo, el alumno podrá incluir
una descripción preliminar de las tareas, una planificación temporal, diagramas de Gantt o
recursos similares que pueda considerar necesarios.

Si hubiera más de una metodología que a juicio del alumno podría ser afín al proyecto,
éstas deberán mencionarse, y justificar la que considera más adecuada (esto puede
considerarse parte de la justificación a la adecuación al problema a resolver).

\newpage

\section{MEDIOS QUE SE PRETENDEN UTILIZAR}

\subsection{Medios Hardware}

El alumno deberá describir los medios hardware que prevé serán necesarios para el
desarrollo del proyecto.


\subsection{Medios Software}

El alumno deberá describir los medios software (lenguajes, entornos de desarrollo,
herramientas de gestión y planificación, etc.) que prevé serán necesarios para el
desarrollo del proyecto

\newpage

\section{REFERENCIAS}

En esta sección se incluirán todas las referencias bibliográficas, ordenadas
alfabéticamente por el primer apellido del primer autor, de las obras de las cuales se
haya realizado alguna cita en los apartados anteriores. Las referencias deberán contener
datos básicos como nombre y apellidos de los autores, título de la obra, evento al que
pertenece, páginas, fecha y lugar de celebración (si se tratara de artículos de congreso),
ISBN, editorial y ciudad (si se tratara de libro), nombre de revista, páginas, volumen y
número (si se tratara de revista), etc.

Se empleará un formato de referencia reconocido en el ámbito académico como
ACM\footnote{http://www.acm.org/sigs/publications/proceedings-templates}\footnote{http://www.cs.ucy.ac.cy/\~{}chryssis/specs/ACM-refguide.pdf}.
Otros formatos aconsejables son, por ejemplo, IEEE, AMA, APA y AMA.

A continuación una sección de «Referencias» con ejemplos de referencias con formato ACM para:

\begin{itemize}
\item Un artículo de revista~\cite{Bow93}.
\item Un informe técnico~\cite{Ding97}.
\item Un libro~\cite{Tavel07}.
\item Un capítulo de libro~\cite{Greiner99}.
\item Un artículo en las actas de un congreso~\cite{Frohlic00}.
\item Para una página web~\cite{Steele04} (con autores conocidos).
\item Para una página web~\cite{Oxygen} (con autores desconocidos).
\end{itemize}

\end{document}


% Local Variables:
% coding: utf-8
% mode: flyspell
% ispell-local-dictionary: "castellano8"
% mode: latex
% End:
