\chapter{Objetivos}
\label{chap:objetivos}

\drop{E}{n} el siguiente capítulo se presentan los objetivos que se marcaron para el desarrollo de
este \acf{TFG} desde la realización del Anteproyecto de forma general y específica. De esta manera,
se determina el alcance del proyecto y los resultados que se esperan tras su implementación.

\section{Objetivo general}

Incentivado por el gran número de accidentes que se producen continuamente por distracciones a la
hora de utilizar los navegadores vía satélite actuales, en este trabajo se desarrollará un sistema
de navegación por satélite que haga uso de la interacción implícita, es decir, un sistema de
navegación que nos avise de las acciones a realizar sin necesidad de prestarle atención.

Dado que la única interacción implícita que ofrecen los navegadores actuales es el sonido, los
ciclistas y motoristas no pueden hacer uso de la navegación por satélite sin incrementar
considerablemente el riesgo de accidente. Esto es debido a que mirar la pantalla del navegador
provoca el 26\% de los accidentes producidos por distracciones \cite{Allianz14}, resulta complicado
oír nuestro dispositivo en ambientes abiertos y el uso de cascos o auriculares esta sancionado en
España \cite{Serrano14}.

Con la finalidad de hacer más segura la navegación para los ciclistas y motoristas, y para facilitar
su uso para los peatones, necesitamos otro tipo de interacción implícita. Así pues, si descartamos
la interacción por medio de la vista por no ser implícita y el oído por lo citado anteriormente,
sólo nos quedan disponibles el gusto, el olfato y el tacto. Y, puesto que la interacción con el
usuario por medio del gusto y el olfato está muy poco desarrollada, sólo nos queda el tacto. Por
lo tanto, haremos uso del tacto por medio de la vibración de nuestros dispositivos.

En resumen, el objetivo del \acf{TFG} es el «desarrollo de un sistema de navegación
  por satélite que se comunique con el usuario por medio de la vibración de alguno de sus
  componentes»

\section{Objetivos específicos}

A partir del objetivo principal descrito en la sección anterior se determinan los objetivos
específicos necesarios para alcanzarlo.

\subsection{Estudio del estado del arte}

En primer lugar será necesario un estudio del estado del arte con la finalidad de identificar las
características principales los navegadores actuales y determinar qué debemos desarrollar. De este
modo, si existiera la posibilidad de añadir un complemento vibratorio a alguna aplicación existente,
no sería necesario desarrollar desde cero una aplicación; bastaría con crear el plugin.

Por otro lado, una evaluación del estado del arte nos proporcionará una visión global de las
aplicaciones de navegación actuales y nos permitirá especificar los requisitos necesarios para
implementar una nueva aplicación.

\subsection{Selección del complemento vibratorio}

Hoy en día existen multitud de complementos vibratorios como relojes, pulseras, móviles, zapatillas,
etc. Por ello, en segundo lugar, será imprescindible buscar entre los diferentes complementos
vibratorios y escoger el que más se ajuste a nuestras necesidades.

Para la elección del complemento se pondrá especial atención en que nos permita activar y desactivar
la vibración cuando estimemos oportuno. Esta característica predominará sobre cualquier otra.

\subsection{Desarrollo de la aplicación}

Una vez terminados los puntos anteriores estaremos en disposición de desarrollar nuestra aplicación.
Para conseguirlo, habrá que tomar algunas decisiones de diseño como:

\begin{itemize}
  \item Elección de la plataforma de desarrollo para smartphone
  \item Elección del proveedor de mapas
  \item Elección de la tecnología de geoposicionamiento
  \item Elección del proveedor de rutas
  \item Elección del proveedor de geocoding\footnote{https://en.wikipedia.org/wiki/Geocoding}
\end{itemize}

Finalmente se procederá a la implementación del sistema con las elecciones tomadas y a la
realización de las pruebas. Estas pruebas determinarán el correcto funcionamiento del sistema y
proporcionarán medidas del rendimiento del mismo.

\section{Limitaciones}

Las limitaciones asociadas al proyecto vendrán dadas por las decisiones tomadas durante el
desarrollo antes descritas. Por ejemplo, cuando seleccionemos una determinada plataforma de
desarrollo para smartphone, tendremos las limitaciones de la plataforma como el lenguaje de
desarrollo. De igual forma, cuando elijamos el proveedor de mapas, la tecnología de posicionamiento,
el proveedor de rutas o el proveedor de geocoding heredaremos la precisión de los servicios que
utilicemos.

% Local Variables:
% TeX-master: "main.tex"
%  coding: utf-8
%  mode: latex
%  mode: flyspell
%  ispell-local-dictionary: "castellano8"
% End:
