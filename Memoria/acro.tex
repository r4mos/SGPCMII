\chapter{Listado de acrónimos}

{\small
\begin{acronym}[XXXXXXXX]
  \acro{TFG}     {Trabajo Fin de Grado}
  \acro{SGNS}    {Sistema Global de Navegación por Satélite}
  \acro{GNSS}    {Global Navigation Satellite System}
  \acro{PDA}     {Personal Digital Assistant}
  \acro{NHTSA}   {National Highway Traffic Safety Administration}
  \acro{SO}      {Sistemas Operativos}
  \acro{API}     {Application Programming Interface}
  \acro{NAVSTAR-GPS} {NAVigation System and Ranging - Global Position System}
  \acro{GPS}     {Global Position System}
  \acro{GLONASS} {Global'naya Navigatsionnaya Sputnikovaya Sistema}
  \acro{BNTS}    {BeiDou/Compass Navigation Test System}
  \acro{QZSS}    {Quasi-Zenith Satellite System} 
  \acro{IRNSS}   {Indian Regional Navigation Satellite System}
  \acro{CES}     {Consumer Electronics Show}
  \acro{HFP}     {Hands-Free Profile}
  \acro{HID}     {Human Interface Device Profile}
  \acro{PAN}     {Personal Area Networking Profile}
  \acro{OSM}     {Open Street Map}
  \acro{SDK}     {Software Development Kit}
  \acro{REST}    {Representational State Transfer}
  \acro{IPO}     {Interacción Persona Ordenador}
  \acro{GUI}     {Graphical User Interface}
  \acro{MAC}     {Media Access Control}
\end{acronym}
}


% \ac{OO}   la primera vez \acf, después \acs
% \acs{OO}  short: OO
% \acf{OO}  full : Object Oriented (OO)
% \acl{OO}  large: Object Oriented
% \acx{OO}         OO (Object Oriented)

% usa \Acro cuando no debe aparecer nunca expandido en el texto

% Local Variables:
% TeX-master: "main.tex"
%  coding: utf-8
%  mode: latex
%  mode: flyspell
%  ispell-local-dictionary: "castellano8"
% End:
