\chapter{Antecedentes}
\label{chap:antecedentes}

\section{Interacción implícita}


\subsection{Interacción implícita en el guiado de personas}


\section{Navegación vía satélite}


\subsection{Tecnologías actuales}


\begin{definitionlist}
  \item[GPS]

  \item[GLONASS]

  \item[Galileo]

\end{definitionlist}


\subsection{Aplicaciones de navegación para smartphone}


\begin{definitionlist}
  \item[Google maps] 

  \item[iOS maps]

  \item[Nokia Here]

  \item[Sygic]

\end{definitionlist}

\section{Plataformas smartphone}

Dese la aparición de la industria dedicada a la telefonía móvil allá por 1983 cuando la compañía
Motorola estrenó el primer móvil de la historia, el Motorola DynaTAC 8000X, se han sucedido una
serie de mejoras continuas hasta ofrecer al usuario la posibilidad de estar continuamente conectado
para llegar a lo que hoy conocemos como smartphone

El éxito de los teléfonos inteligentes o smartphones radica en que ofrecen la posibilidad de llevar
continuamente con nosotros un ordenador en el que se pueden instalar multitud de aplicaciones. Estos
programas tienen finalidades muy dispares, desde aplicaciones relacionadas con el ámbito laboral
como gestores de correo electrónico y editores de texto, hasta aplicaciones de ocio como juegos y
redes sociales.

A lo largo de estos años, los fabricantes se han basado en diferentes \acf{SO} especialmente
diseñados para la telefonía móvil a la hora de gestionar los terminales. Estos \acs{SO} están
enfocados en exprimir las capacidades multimedia e inalámbricas de los dispositivos.

Puesto que cada sistema cuenta con sus propias aplicaciones, los \acs{SO} han ido adquiriendo mayor
importancia de forma que existe una guerra por hacerse por la mayor cuota de mercado.

Este \acs{TFG} pretende aprovechar la popularidad de los smartphone para hacer llegar nuestra
aplicación al mayor número de usuarios. Para ello, tendremos que decidir cuál será la plataforma
sobre la que se desarrollará el proyecto: Android, iOs, Windows Phone o Blackberry.

A pesar de que Symbian de Nokia tenía bastante popularidad hace no demasiados años, queda
completamente descartado del estudio porque en la actualidad posee menos del 0.5\% de cuota de
mercado y está previsto dejar de implantarlo en nuevos terminales a partir de
2016~\cite{Litchfield13}.

\subsection{Android}

Andorid es un \acs{SO} basado en Linux desarrollado por la Open Handset Alliance liderada por
Google. Su lanzamiento tuvo lugar en Octubre de 2008 y, desde entonces, se ha hecho con el
81\%~\cite{Llamas13} de la cuota de mercado de los smartphones. La última versión de este \acs{SO}
 es la 5.0.1 o «Lollipop» y el lenguaje de programación de aplicaciones es Java.

Uno de las principales causas de su éxito es ser un sistema abierto. Esto ha permitido que salgan al
mercado una gran multitud de terminales con características y precios muy distintos que han
conseguido que llegue a la mayoría de los usuarios de smartphone.

\begin{definitionlist}
  \item[Ventajas] Android cuenta con la mayor cuota de mercado y facilidad a la hora de empezar a
    desarrollar y posteriormente probar la aplicación. Además, para distribuir la aplicación por
    medio de la tienda oficial, la Play Store, sólo hay que pagar previamente 25\$.

    Por otro lado, gracias a la variedad de fabricantes, Android posee una gran variedad de
    dispositivos con diferentes características y podremos seleccionar el dispositivo con la
    potencia y características que necesitemos.

  \item[Desventajas] Las principales ventajas también suponen algunos problemas. El hecho de que
    haya diferentes dispositivos con diferente hardware nos complica garantizar el rendimiento
    óptimo de nuestra aplicación en las diferentes configuraciones de pantalla, memoria y
    procesador.

\end{definitionlist}

\subsection{iOS}

Es el \acs{SO} desarrollado por Apple para iPhone aunque posteriormente se portó para otros
dispositivos de la compañía como iPad o iPod Touch. Su lanzamiento tuvo lugar en Junio de 2007 junto
con el primer iPhone y en la actualidad cuenta con un 12.9\%~\cite{Llamas13} de la cuota de
mercado. La última versión de este \acs{SO} es la 8.1.2 y el lenguaje de programación de
aplicaciones es Objetive-C.

Apple iOS ha sido durante mucho tiempo la referencia de los desarrolladores de aplicaciones ya que
fue el primero en incorporar una tienda de aplicaciones: AppStore. Con ello, permitió desarrollar a
terceros aplicaciones para sus dispositivos y mantenerlas bajo el control de Apple.

\begin{definitionlist}
  \item[Ventajas] Puesto que el \acs{SO} está diseñado específicamente para una configuración de
    hardware, iOS permite explotar al máximo sus capacidades y desarrollar aplicaciones para un
    pequeño grupo de dispositivos bastante potentes.

  \item[Desventajas] Apple no permite instalar de forma legal aplicaciones que no hayan sido
    validadas por medio de su AppStore. Si queremos desarrollar aplicaciones deberemos pagar una
    licencia anual de 99\$.

    Por otro lado, los dispositivos de Apple son en general bastante caros y no están al alcance de
    todo el mundo.

\end{definitionlist}

\subsection{Windows Phone}

Es el \acs{SO} desarrollado por Microsoft sucesor de Microsoft Mobile y llamado originalmente Pocket
PC. Fue presentado en 2010 y en la actualidad cuenta con el 3.7\%~\cite{Llamas13} de la cuota de
mercado. La última versión del \acs{SO} es la 8.1 y soporta los lenguajes programación C\# y Visual
Basic .NET.

En 2011 se anunció una alianza con Nokia por la cual se convertirá en el principal \acs{SO} de la
compañía finlandesa. Por ello, a partir de 2016 los teléfonos que antaño utilizaban Symbian
utilizarán Windows Phone.

\begin{definitionlist}
  \item[Ventajas] Microsoft provee de un entorno de trabajo como Visual Studio, con una gran
    cantidad de \acf{API}, que permite que la programación resulte lo más sencilla posible.

  \item[Desventajas] La mayor desventaja de Windows Phone viene dada por su tardía llegada al
    mercado que le ha hecho difícil competir con los otros \acs{SO}. Estos ya tienen bastante cuota
    de mercado y, al no ofrecer grandes saltos de calidad, no ha podido convencer a los usuarios.

    Por otro lado, para publicar aplicaciones en la tienda oficial, es necesario pagar una licencia
    de 75\euro{}.

\end{definitionlist}

\subsection{Blackberry}

Blackberry es un \acs{SO} desarrollado por la empresa canadiense Research In Motion. Tuvo sus
inicios en 1999 incorporando funciones típicas hoy en día como acceso al correo electrónico e
Internet. En la actualidad tiene un 1.7\%~\cite{Llamas13} de cuota de mercado. La última versión de
este \acs{SO} es Blackberry 10 y el lenguaje de programación de las aplicaciones es Java.

Aunque este sistema está orientado especialmente a empresas y profesionales, tuvo un gran éxito
entre los jóvenes gracias a su servicio de mensajería instantánea.

\begin{definitionlist}
  \item[Ventajas] No es necesario pagar para subir aplicaciones a su tienda oficial de aplicaciones
    llamada App World. Sólo es necesario registrarnos como desarrollador y que aprueben nuestra
    aplicación.

    Para determinados tipos de usuario puede considerarse una ventaja que la mayoría de terminales
    dispongan de un teclado físico.

  \item[Desventajas] De igual modo, disponer de un teclado físico limita las posibilidades de
    desarrollo frente a dispositivos con pantallas táctiles.

    Además, su tienda de aplicaciones cuenta con poco apoyo por parte de los desarrolladores y, en
    consecuencia, dispone de muchas menos aplicaciones que la competencia.

\end{definitionlist}

\section{Weareables}



\subsection{Weareables vibratorios para smartphone}


\begin{definitionlist}
  \item[Relojes]

  \item[Pulseras]

  \item[Zapatillas]
 
\end{definitionlist}


\section{Proveedores de mapas y rutas}


\subsection{Google}


\subsection{Open Street Maps}


\section{Proveedores de geocoding}


\subsection{Google}


\subsection{Open Street Maps}

% Local Variables:
% TeX-master: "main.tex"
%  coding: utf-8
%  mode: latex
%  mode: flyspell
%  ispell-local-dictionary: "castellano8"
% End:
