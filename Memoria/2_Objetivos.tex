\chapter{Objetivos}
\label{chap:objetivos}

\drop{E}{n} el presente capítulo se muestran los objetivos que se marcaron para el desarrollo de
este proyecto desde la realización del \emph{anteproyecto} de forma general y específica. De esta
manera, se determina el alcance del proyecto y los resultados que se esperan tras su implementación.

\section{Objetivo general}

Incentivado por el gran número de accidentes que se producen continuamente por distracciones a la
hora de utilizar los navegadores vía satélite actuales, en este trabajo se desarrollará un sistema
de navegación por satélite que haga uso de la \emph{interacción implícita}, es decir, un sistema de
navegación que avise al usuario de las acciones a realizar sin necesidad de prestarle atención.

Dado que la única \emph{interacción implícita} que ofrecen los navegadores actuales es el sonido,
los ciclistas y motoristas no pueden hacer uso de la navegación por satélite sin incrementar
considerablemente el riesgo de accidente. Esto es debido a que mirar la pantalla del navegador
provoca el 26\% de los accidentes producidos por distracciones \cite{Allianz14}, resulta complicado
oír el dispositivo en ambientes abiertos y el uso de cascos o auriculares esta sancionado en
España \cite{Serrano14}.

Con la finalidad de hacer más segura la navegación para los ciclistas y motoristas, y para facilitar
su uso para los peatones, se necesita otro tipo de \emph{interacción implícita}. Así pues, si se
descarta la interacción por medio de la vista por no ser implícita y el oído por lo citado
anteriormente, sólo quedan disponibles el gusto, el olfato y el tacto. Y, puesto que la
interacción con el usuario por medio del gusto y el olfato está muy poco desarrollada, sólo queda el
tacto. Por lo tanto, se hará uso del tacto por medio de la vibración de los dispositivos.

En resumen, el objetivo del \acf{TFG} es el «desarrollo de un sistema de navegación
  por satélite que se comunique con el usuario por medio de la vibración de alguno de sus
  componentes»

\section{Objetivos específicos}

A partir del objetivo principal descrito en la sección anterior se determinan los objetivos
específicos necesarios para alcanzarlo.

\subsection{Desarrollo del sistema de posicionamiento}

Se implementará un sistema que permita conocer la posición del usuario en el globo y
la represente en un mapa. Dicho sistema será el encargado de actualizar la posición del usuario
cuando cambie su localización. Para ello será necesario tomar algunas decisiones como:

\begin{itemize}
  \item Elección de la plataforma de desarrollo
  \item Elección del proveedor de mapas
  \item Elección de tecnología de posicionamiento
\end{itemize}

\subsection{Desarrollo del sistema de navegación}

Se implementará un sistema que permita navegar desde la posición del usuario hasta otra posición
determinada por el nombre de un lugar. Para ello se tomarán algunas decisiones de diseño como:

\begin{itemize}
  \item Elección del proveedor de rutas
  \item Elección del proveedor de \emph{geocoding}\footnote{El geocoding es un proceso mediante el
    cual se obtienen las coordenadas geográficas de un sitio. Para ello basta con introducir una
    descripción, una dirección postal o el nombre del lugar.}
\end{itemize}

\subsection{Desarrollo del sistema de reproducción de audio}

Se implementará un sistema de reproducción de audio que permita escuchar las instrucciones
necesarias para seguir una ruta. Para ello será necesario seleccionar un proveedor de \emph{text to
  speech}\footnote{Un servicio text to speech proporciona el audio de un texto introducido}.

\subsection{Desarrollo del sistema de comunicación con otros dispositivos}

Se implementará un sistema de comunicación que permita el paso de mensajes entre los diferentes
dispositivos del sistema. Para ello será necesario elegir la tecnología o tecnologías (Wi-Fi,
Bluetooth, etc) sobre la que se implementará la arquitectura cliente-servidor.

\subsection{Desarrollo del sistema vibratorio}

Se implementará un sistema vibratorio que, con ambos dispositivos, codifique todas las posibles
acciones que el usuario pueda realizar. Para ello, hará uso de diferentes frecuencias de vibración en los dispositivos de modo que la codificación resulte intuitiva.

\section{Limitaciones}

Las limitaciones asociadas al proyecto vendrán dadas por las decisiones tomadas durante el
desarrollo antes descritas. Por ejemplo, cuando se seleccione una determinada plataforma de
desarrollo para \emph{smartphone}, se tendrán las limitaciones de la plataforma como el lenguaje de
desarrollo. De igual forma, cuando se elija el proveedor de mapas, la tecnología de posicionamiento,
el proveedor de rutas o el proveedor de \emph{geocoding} se heredará la precisión de los servicios
que se utilicen.

% Local Variables:
% TeX-master: "main.tex"
%  coding: utf-8
%  mode: latex
%  mode: flyspell
%  ispell-local-dictionary: "castellano8"
% End:
