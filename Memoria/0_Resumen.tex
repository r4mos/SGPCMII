\chapter{Resumen}

No hace demasiado tiempo, resultaba increíble llevar en el bolsillo una herramienta que permitiera
conocer la localización geográfica exacta del usuario. Mucho menos creíble resultaba imaginar que
esa herramienta otorgaría la capacidad de estar siempre conectado a Internet. No obstante, hoy en
día esa herramienta existe y se llama \emph{smartphone}.

Con la reciente popularización de los \emph{smartphones} han surgido una larga serie de complementos
\emph{wearables} o «llevables» que se comunican con los \emph{smartphones} y proporcionan una nueva
forma de interacción con el dispositivo.

En el presente proyecto se pone de manifiesto uno de los problemas más importantes relacionados con
el uso de los \emph{smartphones} para la navegación por satélite: la facilidad de, cuando se conduce
un vehículo, ocasionar una distracción y en numerosas ocasiones un accidente. Para solucionarlo se
plantea una nueva forma de interacción en la que el sistema se encargue de avisar al usuario en el
momento que tenga que realizar alguna acción por medio de \emph{wearables}. De este modo se evita
que el usuario tenga que mirar la pantalla y se previenen ese tipo de accidentes.

La aplicación desarrollada a lo largo del proyecto se llama \emph{Naviganto} y ofrece al usuario la
posibilidad de navegar por satélite guiado por las vibraciones de dos dispositivos: uno colocado en
la parte izquierda del cuerpo y el otro situado en la parte derecha. Además, \emph{Naviganto}
proporciona al usuario las interacciones comunes de las aplicaciones de navegación como la pantalla
o el sonido.

% Local Variables:
% TeX-master: "main.tex"
%  coding: utf-8
%  mode: latex
%  mode: flyspell
%  ispell-local-dictionary: "castellano8"
% End:
