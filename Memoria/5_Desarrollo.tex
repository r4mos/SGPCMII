\chapter{Desarrollo del proyecto}
\label{chap:desarrollo}

\drop{E}{n} este capítulo se describe el proceso de desarrollo de nuestro sistema de navegación por
satélite. Se empieza enumerando los requisitos originales del sistema y se explica cómo se
realizarán las pruebas. Más tarde se explica iteración a iteración las decisiones tomadas, los
prototipos desarrollados y las pruebas realizadas sobre ellos.

\section{Especificación de requisitos}

Como se comentó anteriormente (ver sección~\ref{sec:metodologia}) los requisitos del sistema se
han ido detallando a lo largo del desarrollo del proyecto ya que, originalmente, sólo teníamos una
idea muy general del sistema a desarrollar. A continuación se muestran estos requisitos generales
sobre los que partimos:

\begin{itemize}
  \item El sistema debe poder guiar a peatones, ciclistas y motoristas.
  \item El sistema debe desplegarse sobre alguna plataforma de smartphone.
  \item La interacción con el sistema debe desarrollarse de forma implícita y ser válida para
    peatones, ciclistas y motoristas.
  \item En caso de necesitar complementos para la interacción, deben estar disponibles en el
    mercado.
  \item El sistema debe implementar las características básicas del resto de aplicaciones del
    mercado como mostrar la posición en el mapa, rotar el mapa en función de nuestra orientación o
    visualizar la ruta que se va a seguir.
\end{itemize}

\section{Pruebas}

Para describir las pruebas realizadas sobre los diferentes prototipos en este documento se ha
utilizado el patrón «Given-When-Then». Este patrón divide el proceso de ejección en 3 etapas:

\begin{itemize}
  \item \textbf{Given} (Dado): Condiciones previas sobre las que se producen los eventos.
  \item \textbf{When} (Cuando): Operaciones específicas que se producen.
  \item \textbf{Then} (Entonces): Resultados esperados.
\end{itemize}

\section{Proceso de desarrollo}

A continuación se indican las iteraciones realizadas durante el desarrollo del \acs{TFG} detallando
los objetivos, el diseño, la implementación y las pruebas de cada una de ellas.

\subsection{Iteración 1: Mostrar mapa}

En la primera iteración nos marcamos como objetivo mostrar una posición cualquiera en el mapa por
medio de una aplicación de smartphone.

\subsubsection{Diseño}

Para realizar esta primera aproximación tendremos que tomar tres importantes decisiones de diseño:

\begin{itemize}
  \item Elegir la plataforma de smartphone sobre la que desplegar nuestra aplicación.
  \item Elegir el proveedor de mapas que nos proporcionará las imágenes a mostrar.
  \item Elegir la versión objetivo de la plataforma de desarrollo.
\end{itemize}

En primer lugar, tras revisar las diferentes plataformas disponibles para smartphone (ver
sección~\ref{sec:plataformas}) se estipuló que la mejor alternativa es Android. Por un lado, Android
posee de la mayor cuota de mercado (81\%) y dispone de una gran variedad de dispositivos en una
amplia gama de precios. Por otro lado, Android es un \acs{SO} que se integra fácilmente con la
plataforma de complementos Android Wear (ver sección~\ref{sec:wearables}) que es la única que nos
permite manipular el vibrador del wearable a voluntad.

En segundo lugar, entre los proveedores de mapas existentes (ver sección~\ref{sec:proveedores}) se
seleccionó Open Street Map. Se consideró la mejor elección porque nos provee de mapas de gran
calidad completamente gratuitos. Además, si encontrásemos cualquier tipo de error en los mapas
suministrados, podríamos corregirlo porque es un proyecto colaborativo. Para simplificar el uso de
\acs{OSM} se utilizó la librería Osmdroid tal y como se dijo en la
sección~\ref{sec:herramientasSoftware}.

En tercer y último lugar, se seleccionó como objetivo del desarrollo la \acs{API} de Android 21,
también conocida como Android 5.0 (Lollipop), por ser la más nueva y poseer el mayor número de
funcionalidades. De todos modos, se aseguró la compatibilidad desde la \acs{API} de Android 8,
también llamada Android 2.2 (Froyo), por medio de la librería Android support.

\subsubsection{Implementación}

Para visualizar los mapas de \acs{OSM} con la ayuda de la librería Osmdroid basta con añadir una
vista del tipo \texttt{org.osmdroid.views.MapView} al layout de nuestra aplicación (ver
listado~\ref{code:layoutMapView}) y seleccionar el punto del mapa que deseamos centrar en la
pantalla (ver listado~\ref{code:activityMapView}) por medio de sus coordenadas geográficas.

\begin{listing}[
  float=ht,
  language = xml,
  caption  = {Ejemplo de layout usando \texttt{org.osmdroid.views.MapView}},
  label    = code:layoutMapView]
<?xml version="1.0" encoding="utf-8"?>
<LinearLayout xmlns:android="http://schemas.android.com/apk/res/android"
        xmlns:tools="http://schemas.android.com/tools"
        android:orientation="vertical" 
        android:layout_width="fill_parent"
        android:layout_height="fill_parent">
        
        <org.osmdroid.views.MapView android:id="@+id/map"
                android:layout_width="fill_parent" 
                android:layout_height="fill_parent" />
                
</LinearLayout>
\end{listing}

\begin{listing}[
  float=ht,
  language = java,
  caption  = {Ejemplo de activity mostrando un punto de un mapa en específico},
  label    = code:activityMapView]
public class MainActivity extends Activity {
    @Override public void onCreate(Bundle savedInstanceState) {
        super.onCreate(savedInstanceState);
        setContentView(R.layout.main);

        MapView map = (MapView)findViewById(R.id.map);
        map.setMultiTouchControls(true);
        map.getController().setZoom(18);

        GeoPoint startPoint = new GeoPoint(39.40540171, -3.12204771);
        map.getController().setCenter(startPoint);
    }
}
\end{listing}

Por muy sencillo que pueda parecer, durante la implementación detectamos un problema: el mapa no se
centraba en la posición seleccionada. Dejaba dicha posición en la esquina superior izquierda y no en
el centro de la pantalla. Estudiando el problema descubrimos que se trataba de un bug
documentado~\footnote{https://github.com/osmdroid/osmdroid/issues/22\#issuecomment-43092313} de
Osmdroid y que se resolvería en la próxima versión de la librería.

Para paliar dicho problema, y siguiendo con las instrucciones detalladas en el bug, procedimos a
implementar un método que lo resolviera (ver listado~\ref{code:centerMap}).

\begin{listing}[
  float=ht,
  language = java,
  caption  = {Método utilizado para centrar el mapa en cualquier posición},
  label    = code:centerMap]
private void centerMap(final GeoPoint loc) {
    mMap.getViewTreeObserver().addOnGlobalLayoutListener(
            new ViewTreeObserver.OnGlobalLayoutListener() {
        @Override
        public void onGlobalLayout() {
            mMap.getViewTreeObserver().removeGlobalOnLayoutListener(this);
            mMap.getController().setCenter(loc);
        }
    });
}
\end{listing}

\subsubsection{Pruebas}

Para corroborar el correcto funcionamiento del primer prototipo se realizaron dos grupos de pruebas.

\newpage
\begin{itemize}
  \item Para determinar la correcta carga de los mapas en diferentes localizaciones:

  \begin{tabular}{p{.15\textwidth}p{.75\textwidth}}
    \hline
    \textbf{Dado}     & Diferentes coordenadas geográficas \\
    \textbf{Cuando}   & Ejecutamos la aplicación \\
    \textbf{Entonces} & Se muestra dicha ubicación centrada en un mapa \\
    \hline
  \end{tabular}

  \item Para determinar el correcto funcionamiento en diferentes versiones de Android:

  \begin{tabular}{p{.15\textwidth}p{.75\textwidth}}
    \hline 
    \textbf{Dado} & Diferentes dispositivos con distintas versiones de Android (ver
    sección~\ref{sec:herramientasHardware}) \\
    \textbf{Cuando} & Ejecutamos la aplicación \\ 
    \textbf{Entonces} & Se muestra la ubicación correspondiente centrada en un mapa \\ 
    \hline
  \end{tabular}
\end{itemize}

\subsection{Iteración 2: Mostrar posición actual en el mapa}
\subsubsection{Diseño}
\subsubsection{Implementación}
Elección de tecnología de posicionamiento
\subsubsection{Pruebas}
\begin{tabular}{p{.15\textwidth}p{.75\textwidth}}
  \hline
  \textbf{Dado}     & ... \\
  \textbf{Cuando}   & ... \\
  \textbf{Entonces} & ... \\
  \hline
\end{tabular}

\subsection{Iteración 3: Orientar mapa}
\subsubsection{Diseño}
\subsubsection{Implementación}
Uso de sensores
\subsubsection{Pruebas}
\begin{tabular}{p{.15\textwidth}p{.75\textwidth}}
  \hline
  \textbf{Dado}     & ... \\
  \textbf{Cuando}   & ... \\
  \textbf{Entonces} & ... \\
  \hline
\end{tabular}

\subsection{Iteración 4: Crear y mostrar ruta}
\subsubsection{Diseño}
\subsubsection{Implementación}
Para peatones ciclistas y motoristas por separado.
Problemas con las rutas rectas
\subsubsection{Pruebas}
\begin{tabular}{p{.15\textwidth}p{.75\textwidth}}
  \hline
  \textbf{Dado}     & ... \\
  \textbf{Cuando}   & ... \\
  \textbf{Entonces} & ... \\
  \hline
\end{tabular}

\subsection{Iteración 5: Navegar por ruta}
\subsubsection{Diseño}
\subsubsection{Implementación}
\subsubsection{Pruebas}
\begin{tabular}{p{.15\textwidth}p{.75\textwidth}}
  \hline
  \textbf{Dado}     & ... \\
  \textbf{Cuando}   & ... \\
  \textbf{Entonces} & ... \\
  \hline
\end{tabular}

\subsection{Iteración 6: Selector de destino}
\subsubsection{Diseño}
\subsubsection{Implementación}
Problemas al buscar por número de casa
\subsubsection{Pruebas}
\begin{tabular}{p{.15\textwidth}p{.75\textwidth}}
  \hline
  \textbf{Dado}     & ... \\
  \textbf{Cuando}   & ... \\
  \textbf{Entonces} & ... \\
  \hline
\end{tabular}

\subsection{Iteración 7: Avisos por pantalla}
\subsubsection{Diseño}
\subsubsection{Implementación}
\subsubsection{Pruebas}
\begin{tabular}{p{.15\textwidth}p{.75\textwidth}}
  \hline
  \textbf{Dado}     & ... \\
  \textbf{Cuando}   & ... \\
  \textbf{Entonces} & ... \\
  \hline
\end{tabular}

\subsection{Iteración 8: Avisos sonoros}
\subsubsection{Diseño}
\subsubsection{Implementación}
Por qué no avisos grabados...
\subsubsection{Pruebas}
\begin{tabular}{p{.15\textwidth}p{.75\textwidth}}
  \hline
  \textbf{Dado}     & ... \\
  \textbf{Cuando}   & ... \\
  \textbf{Entonces} & ... \\
  \hline
\end{tabular}

\subsection{Iteración 9: Avisos vibratorios}
\subsubsection{Diseño}
\subsubsection{Implementación}
Elección de complemento... puesto que es una barrera de entrada el smartwatch se desarrolla para
también para otros móviles.
Codificación de la vibración (rotondas y tal)
Hablar del paso de mensajes
Problemas con el paso de mensajes por agrupar pensajes

\subsubsection{Pruebas}
\begin{tabular}{p{.15\textwidth}p{.75\textwidth}}
  \hline
  \textbf{Dado}     & ... \\
  \textbf{Cuando}   & ... \\
  \textbf{Entonces} & ... \\
  \hline
\end{tabular}

% Local Variables:
% TeX-master: "main.tex"
%  coding: utf-8
%  mode: latex
%  mode: flyspell
%  ispell-local-dictionary: "castellano8"
% End:
